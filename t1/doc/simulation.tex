\section{Simulation Analysis}
\label{sec:simulation}
\paragraph{}

\par This simulation was run through a software called NGSpice. From it, the results obtained were the following. 
\begin{table}[H]
  \centering
  \begin{tabular}{|c|c|}
    \hline    
    {\bf Name} & {\bf Value [A or V]} \\ \hline
    @vx[i] & 9.804367e-04\\ \hline
@hc[i] & 3.412015e-05\\ \hline
@va[i] & -2.44092e-04\\ \hline
@gb[i] & -2.55448e-04\\ \hline
@id[current] & 1.014557e-03\\ \hline
@r1[i] & -2.44092e-04\\ \hline
@r2[i] & -2.55448e-04\\ \hline
@r3[i] & -1.13559e-05\\ \hline
@r4[i] & -1.22453e-03\\ \hline
@r5[i] & -1.27000e-03\\ \hline
@r6[i] & 9.804367e-04\\ \hline
@r7[i] & 9.804367e-04\\ \hline
n1 & 2.524677e-01\\ \hline
n2 & -5.18183e-01\\ \hline
n3 & 3.572797e-02\\ \hline
n4 & -4.90449e+00\\ \hline
n5 & 4.000301e+00\\ \hline
n6 & -6.93514e+00\\ \hline
n7 & -7.93124e+00\\ \hline
n8 & -6.93514e+00\\ \hline

  \end{tabular}
  \caption{Operating point. A variable preceded by @ is of type {\em current}
    and expressed in Ampere; other variables are of type {\it voltage} and expressed in
    Volt.}
  \label{tab:op}
\end{table}

\par Such results are sufficient for a thourough analysis of the entire circuit. 
\par Firstly, by noting that $V_b$ is the imposed voltage between nodes 2 and 5, one concludes that $V_b$ is then given by $V_2$ given that node 5 is attached to ground. Likewise, and given that $V_c$ is the imposed voltage between nodes 5 and 8 and since node 5 is connected to ground, $V_c$ is then given by negative $V_8$.

\begin{equation}
	V_b=V_{2}
\end{equation}
\begin{equation}
	V_c=-V_{8}
\end{equation}

\par Secondly, by noting that currents passing through $R_1$, $R_2$ and $R_6$ are equal to $I_a$, $I_b$ and $I_c$ respectively, one can directly apply Ohms Law to determine these currents. While $I_b$ and $I_c$ could be determined through direct application of the law, it could also be determined by using the relation described in equations 3 and 4. 
\par From such calculations, the following results were obtained. 
\begin{table}[H]
    \centering
    \begin{tabular}{|c|c|}
    \hline
        $I_a$ & 2.40136e-04\\ \hline
        $I_b$ & 2.51245e-04\\ \hline
        $I_c$ & 9.76838e-04\\ \hline
        $V_b$ & 3.4437e-02\\ \hline
        $V_c$ & 7.979210e+00\\ \hline
    \end{tabular}
    \caption{Table of results for the simulation in A and V}
\end{table}



