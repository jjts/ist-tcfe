\section{Simulation Analysis}
\label{sec:simulation}

\subsection{Operating Point Analysis}


The Table~\ref{tab:op} shows the simulated operating point results for the circuit described in Figure~\ref{fig:circuito}.

\begin{figure}[h] \centering
\includegraphics[width=0.6\linewidth]{malhaC.pdf}
\caption{C Mesh with an adicional voltage source} %mudar legendaaaaaaa
\label{fig:malhaC}
\end{figure}

In this simulation is important to explain the creation of an auxiliary voltage $V_b$ (with a voltage equal to $0V$) that was put between N6 and R7 as shown in Figure~\ref{fig:malhaC}. Consequently, this led to the appearance of a node that we designated by N8 that has the same voltage as N6.

This was necessary because of Ngspice software requirements.

By observing the Table~\ref{tab:op},Table~\ref{tab:nodal} and Table~\ref{tab:mesh} that the simulated results are the same as the theoretical results.%são mesmo mesmo iguais?

\begin{table}[h]
  \centering
  \begin{tabular}{|l|r|}
    \hline    
    {\bf Name} & {\bf Value [A or V]} \\ \hline
    @vx[i] & 9.804367e-04\\ \hline
@hc[i] & 3.412015e-05\\ \hline
@va[i] & -2.44092e-04\\ \hline
@gb[i] & -2.55448e-04\\ \hline
@id[current] & 1.014557e-03\\ \hline
@r1[i] & -2.44092e-04\\ \hline
@r2[i] & -2.55448e-04\\ \hline
@r3[i] & -1.13559e-05\\ \hline
@r4[i] & -1.22453e-03\\ \hline
@r5[i] & -1.27000e-03\\ \hline
@r6[i] & 9.804367e-04\\ \hline
@r7[i] & 9.804367e-04\\ \hline
n1 & 2.524677e-01\\ \hline
n2 & -5.18183e-01\\ \hline
n3 & 3.572797e-02\\ \hline
n4 & -4.90449e+00\\ \hline
n5 & 4.000301e+00\\ \hline
n6 & -6.93514e+00\\ \hline
n7 & -7.93124e+00\\ \hline
n8 & -6.93514e+00\\ \hline

  \end{tabular}
  \caption{Operating point. A variable preceded by @ is of type {\em current}
    and expressed in Ampere; other variables are of type {\it voltage} and expressed in
    Volt.}
  \label{tab:op}
\end{table}

However, was also calculated the relative errors made in order to understand the accuracy of the results. The results of this procedure are presented in Table~\ref{tab:erros}

\begin{table}[ht] \centering
\begin{tabular}{|l|c|}
\hline
{\bf Name} & {\bf Relative Error} \\ \hline
{V1}               & 0               \\ \hline
{V2}               & 0               \\ \hline
{V3}               & 0               \\ \hline
{V4}               & 0               \\ \hline
{V5}               & 0                       \\ \hline
{V6}               & 0                       \\ \hline
{V7}               & 0               \\ \hline
{IA}               & 0               \\ \hline
{IB}               & 0               \\ \hline
{IC}               & 0                       \\ \hline
\end{tabular}
\caption{Relative Errors between Octave and NgSpice results}
  \label{tab:erros}
\end{table}



