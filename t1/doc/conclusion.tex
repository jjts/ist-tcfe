\section{Conclusion}
\label{sec:conclusion}
\paragraph{}
\par In this cirtcuit in specific there were only linear components, which enabled us to use the Kirchoff laws and subsequently the mesh and nodal analysis. Another consequence of this linearity is the superposition theroem, which is the basis for the mesh method, given it validates the concept of having more than one current going through a component, which can be viewed independently and allow for a method to succesfully discover all the unkowns in the circuit.
\par The nodal analysis is, as previously mentioned, the one that is easier to automize and is therefore included in \textit{NGSpice}, the simulator used in this activity. It was, therefore, possible to draw conclusions that validate this method that allows for the proper examination of the board.
\par It is now possible to affirm that both theoretical methods are viable due to the fact that similar results were obtained when calculated by a simulator. The previously mentioned linearity of the system accompanied by the success of the \textit{NGSpice} simulation prove this thesis.
\par Concluding, it is possible to analyze this circuit not only through simulation, but also theoretically using the Kirchoff Laws and the nodal and mesh methods because of circuits linearity.
