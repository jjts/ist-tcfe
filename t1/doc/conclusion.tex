\section{Conclusion}
\label{sec:conclusion}

In this laboratory assignment the objective of analysing a simple circuit has been
achieved. The analysis of the circuit was done both theoretically, 
using  the Ocatve maths tool to solve the systems of equations obtained by using the mesh 
and node analysis, and by circuit simulation, using Ngspice. The simulation results perfectly 
matched the theoretical results. For example, if we look at the results tables for ngspice and octave we may note that the 
following relations hold:
\begin{equation*}
    y_1 = -r1[i]
\end{equation*}
\begin{equation*}
    y_2 = r2[i]
\end{equation*}
\begin{equation*}
    y_3 = r6[i] = r7[i]
\end{equation*}
\begin{equation*}
    C_i = 1/r[i]
\end{equation*}
which is to be expected.
The reason for this perfect match is that the analysed circuit is linear,
 so that any method for analysing consists in solving a linear system of equations. 