\section{Conclusion}
\label{sec:conclusion}
\paragraph{}
\par The circuit studied is linear because it only has linear components. The Kirchhoff Laws only work because of this linearity, as do the two methods that we used, which are the circuit mesh and nodal analysis methods. The all-linear relationships have as consequence the validation of the Superposition Theorem. The superposition means that makes sense having multiple mesh currents circulating in only one element of the circuit, and multiple currents enables the use of the mesh currents as our independent variables, using all this to solve the circuit. 
\par As for the second, the nodal analysis method, it's based on the KCL and this method is included in the \textit{NGSpice} simulator that we used. This one is based on the thought of tension in the nodes, referring to the potential differences between 2 nodes of a circuit.
\par As we can see, the reason why both the methods and the \textit{NGSpice} results accomplished are the same is because they are all based on the Kirchhoff's circuit Laws and the linearity of the circuit that allows us to use for example the Superposition Theorem.
\par In conculsion, the simulation and the theoretical analysis have similar results due to the linearity of the circuit.
